% $Header: /cvsroot/latex-beamer/latex-beamer/examples/beamerexample1.tex,v 1.47 2004/11/04 15:43:51 tantau Exp $

\documentclass{beamer}
%\documentclass{article}
%\usepackage[envcountsect]{beamerarticle}

% Do NOT take this file as a template for your own talks. Use a file
% in the directory solutions instead. They are much better suited.

% Try the class options [notes], [notes=only], [trans], [handout],
% [red], [compress], [draft] and see what happens!

% Copyright 2003 by Till Tantau <tantau@users.sourceforge.net>.
%
% This program can be redistributed and/or modified under the terms
% of the LaTeX Project Public License Distributed from CTAN
% archives in directory macros/latex/base/lppl.txt.

% For a green structure color use:
%\colorlet{structure}{green!50!black}

\mode<article> % only for the article version
{
  \usepackage{fullpage}
  \usepackage{hyperref}
}


\mode<presentation>
{
  \setbeamertemplate{background canvas}[vertical shading][bottom=red!10,top=blue!10]

  \usetheme{Warsaw}
  \usefonttheme[onlysmall]{structurebold}
}

%\setbeamercolor{math text}{fg=green!50!black}
%\setbeamercolor{normal text in math text}{parent=math text}

\usepackage{pgf,pgfarrows,pgfnodes,pgfautomata,pgfheaps,pgfshade}
\usepackage{amsmath,amssymb}
\usepackage[latin1]{inputenc}
\usepackage{colortbl}
\usepackage[english]{babel}

%\usepackage{lmodern}
%\usepackage[T1]{fontenc} 

\usepackage{times}

\setbeamercovered{dynamic}

%% Titel etc
\title[System Design \& Implementation]{Taskserver Design}
\author[Tepelmann, Noe, Lapp]{Jan Tepelmann \and Marcel Noe \and Moritz Lapp}
\institute[Universit�t Karlsruhe (TH)]{
        \inst{} SDI Group $\log_2(256)$ \\
                Universit�t Karlsruhe (TH)}
\date[SDI 2008]{System Design \& Implementation, 2008}
\subject{Operating System Design}


%% Hauptseite

\begin{document}
\frame{\titlepage}

\section<presentation>*{Outline}

\begin{frame}
  \frametitle{Outline}
  \tableofcontents[part=1,pausesections]
\end{frame}

\AtBeginSubsection[]
{
  \begin{frame}<beamer>
    \frametitle{Outline}
    \tableofcontents[current,currentsubsection]
  \end{frame}
}

\part<presentation>{Main Talk}

\section{Design}

\subsection{System Components}

\begin{frame}
    \frametitle{System Components}

    \begin{pgfpicture}{0cm}{0cm}{20cm}{7.5cm}
        \color{red}
        \pgfnodebox{L4}[stroke]{\pgfxy(5,7)}{\color{black}L4 Microkernel}{15pt}{2pt}
        \color{green}
        \pgfnodebox{Sigma 0}[stroke]{\pgfxy(5,6)}{\color{black}Sigma 0}{15pt}{2pt}
        \color{blue}
        \pgfnodebox{Syscall Server}[stroke]{\pgfxy(7,5)}{\color{black}Syscall Server}{15pt}{2pt}
        \color{black}
        \pgfnodebox{Anonymous Memory Provider}[stroke]{\pgfxy(2,5)}{Anonymous Memory Provider}{15pt}{2pt}
        \pgfnodebox{Data Space Providers}[stroke]{\pgfxy(2,4)}{Data Space Providers}{15pt}{2pt}
        \pgfnodebox{ELF Loader}[stroke]{\pgfxy(10,3.5)}{ELF Loader}{15pt}{2pt}
        \pgfnodebox{Fileserver}[stroke]{\pgfxy(10,2.5)}{Fileserver}{15pt}{2pt}
        \pgfnodebox{Taskserver}[stroke]{\pgfxy(5,1)}{Taskserver}{15pt}{2pt}

%        \pgfsetendarrow{\pgfarrowto}
%        \pgfnodeconncurve{Syscall Server}{Taskserver}{-90}{90}{.5cm}{.5cm}

    \end{pgfpicture}
\end{frame}

\subsection{Sawmill Inspired Data Spaces}

\begin{frame}
    \frametitle{Sawmill Inspired Data Spaces}

     \begin{itemize}
        \item Every address space has got it's own managing thread, called \emph{region mapper} 
        \item \emph{region mapper} resides at the end of user address space, just below kernel
        \item \emph{region mapper} holds mapping between \emph{VM Region} and \emph{Data Space Provider}

        \begin{pgfpicture}{0cm}{0cm}{7cm}{4cm}
            \color{black}
            \pgfnodebox{User Address Space}[fill]{\pgfxy(2,3)}{}{50pt}{5pt}
            \color{blue}
            \pgfnodebox{Stack}[fill]{\pgfxy(4,3)}{}{10pt}{5pt}
            \color{green}
            \pgfnodebox{Region Mapper}[fill]{\pgfxy(5,3)}{}{18pt}{5pt}
            \color{red}
            \pgfnodebox{Kernel Space}[fill]{\pgfxy(6,3)}{}{25pt}{5pt}

            \color{black}
            \pgfnodebox{User Label}[virtual]{\pgfxy(2,1.5)}{User Space}{50pt}{5pt}
            \pgfsetendarrow{\pgfarrowto}
            \pgfnodeconnline{User Label}{User Address Space}

            \color{blue}
            \pgfnodebox{Stack Label}[virtual]{\pgfxy(3.5,1)}{Main Stack}{50pt}{5pt}
            \pgfsetendarrow{\pgfarrowto}
            \pgfnodeconncurve{Stack Label}{Stack}{90}{-90}{.5cm}{.5cm}

            \color{green}
            \pgfnodebox{Region Label}[virtual]{\pgfxy(5,1.5)}{Region Mapper}{50pt}{5pt}
            \pgfsetendarrow{\pgfarrowto}
            \pgfnodeconncurve{Region Label}{Region Mapper}{90}{-90}{.5cm}{.5cm}

            \color{red}
            \pgfnodebox{Kernel Label}[virtual]{\pgfxy(7,1)}{Kernel Space}{50pt}{5pt}
            \pgfsetendarrow{\pgfarrowto}
            \pgfnodeconncurve{Kernel Label}{Kernel Space}{90}{-90}{.5cm}{.5cm}


        \end{pgfpicture}

     \end{itemize}

\end{frame}

\subsection{Stack Positioning}

\begin{frame}
    \frametitle{Stack Positioning}

     \begin{itemize}
        \item \emph{Main program stack} is created just below \emph{Region Mapper}, growing down, towards \emph{heap}
        \item For every additional thread, \emph{stack space} is allocated from \emph{heap}, surrounded by \emph{read only pages} to detect overflow 
        \item \emph{Thread stacks} are created by \emph{region mapper} 

        \begin{pgfpicture}{0cm}{0cm}{7cm}{4cm}
            \color{black}
            \pgfnodebox{User Address Space}[fill]{\pgfxy(2,3)}{}{20pt}{5pt}

            \color{blue}
            \pgfnodebox{Thread Stack}[fill]{\pgfxy(2.6,3)}{}{6pt}{5pt}

            \color{black}
            \pgfnodebox{User Address Space 2}[fill]{\pgfxy(3.5,3)}{}{20pt}{5pt}

            \color{blue}
            \pgfnodebox{Stack}[fill]{\pgfxy(4,3)}{}{10pt}{5pt}
            \color{green}
            \pgfnodebox{Region Mapper}[fill]{\pgfxy(5,3)}{}{18pt}{5pt}
            \color{red}
            \pgfnodebox{Kernel Space}[fill]{\pgfxy(6,3)}{}{25pt}{5pt}

            \color{blue}
            \pgfnodebox{Thread Stack Label}[virtual]{\pgfxy(1,2)}{Thread Stack}{50pt}{5pt}
            \pgfsetendarrow{\pgfarrowto}
            \pgfnodeconncurve{Thread Stack Label}{Thread Stack}{90}{-90}{.5cm}{.5cm}

            \color{black}
            \pgfnodebox{User Label}[virtual]{\pgfxy(2.5,1.5)}{User Space}{50pt}{5pt}
            \pgfsetendarrow{\pgfarrowto}
            \pgfnodeconnline{User Label}{User Address Space}
            \pgfsetendarrow{\pgfarrowto}
            \pgfnodeconnline{User Label}{User Address Space 2}

            \color{blue}
            \pgfnodebox{Stack Label}[virtual]{\pgfxy(3.5,1)}{Main Stack}{50pt}{5pt}
            \pgfsetendarrow{\pgfarrowto}
            \pgfnodeconncurve{Stack Label}{Stack}{90}{-90}{.5cm}{.5cm}

            \color{green}
            \pgfnodebox{Region Label}[virtual]{\pgfxy(5,1.5)}{Region Mapper}{50pt}{5pt}
            \pgfsetendarrow{\pgfarrowto}
            \pgfnodeconncurve{Region Label}{Region Mapper}{90}{-90}{.5cm}{.5cm}

            \color{red}
            \pgfnodebox{Kernel Label}[virtual]{\pgfxy(7,1)}{Kernel Space}{50pt}{5pt}
            \pgfsetendarrow{\pgfarrowto}
            \pgfnodeconncurve{Kernel Label}{Kernel Space}{90}{-90}{.5cm}{.5cm}


        \end{pgfpicture}


     \end{itemize}


    
\end{frame}

\section{Interface}
\subsection{Process management}

\begin{frame}
    \frametitle{Process management}

    \begin{itemize}
        \item \emph{startTask()}
        \item \emph{fork()}
        \item \emph{exec()}
        \item \emph{kill()}
        \item \emph{waitTid()}
        \item \emph{createThread()}
        \item \emph{getThreadStatus()}
    \end{itemize}
\end{frame}


\begin{frame}
    \frametitle{startTask}

    \texttt{L4\_ThreadId\_t startTask(in String path, in String args);}

    \begin{itemize}
        \item \emph{Task server} creates a new \emph{region mapper} inside its new \emph{address space} using the \emph{syscall server}
        \item \emph{Task server} configures the \emph{address space}
        \item \emph{Task server} sends message to \emph{Region mapper}, telling it the path of the image to load
        \item \emph{Region mapper} asks \emph{ELF-Loader} (or \emph{PE-Loader} or whatever) to map image into its \emph{address space} 
        \item \emph{Region mapper} starts mapped program inside new thread 
    \end{itemize}
\end{frame}

\begin{frame}
    \frametitle{fork}

    \texttt{L4\_ThreadId\_t fork();}

    \begin{itemize}
        \item \emph{Task server} asks \emph{memory server} to create a new \emph{address space} 
        \item \emph{Task server} creates a new \emph{region mapper} inside new \emph{address space}
        \item \emph{Task server} asks \emph{region mappers} to map old \emph{User space} and \emph{Stack} into new \emph{address space}
        \item \emph{Task server} sends message to \emph{region mapper}
        \item \emph{Region mapper} resumes operation in new address space
    \end{itemize}

\end{frame}

\begin{frame}
    \frametitle{exec}

    \texttt{L4\_ThreadId\_t exec(in String path, in String args);}


    \begin{itemize}
        \item \emph{Task server} kills alll \emph{threads} inside address space except \emph{region mapper}
        \item \emph{Task server} sends message to \emph{Region mapper}, telling it the path of the image to load
        \item \emph{Region mapper} asks \emph{ELF-Loader} (or \emph{PE-Loader} or whatever) to map image into its \emph{address space} 
        \item \emph{Region mapper} starts mapped program inside new thread 
    \end{itemize}

\end{frame}

\begin{frame}
    \frametitle{kill}

    \texttt{Void kill(in L4\_ThreadId\_t tid);}

    \begin{itemize}
        \item If \emph{TID} is a region mapper: Kill all \emph{threads} in \emph{address space} 
        \item Else kill \emph{thread} specified by \emph{TID}
        \item \emph{Suicidal tendencies}: Every thread has to kill itself at the end
    \end{itemize}

\end{frame}

\begin{frame}
    \frametitle{waitTid}

    \texttt{L4\_Word\_t waitTid(in L4\_ThreadId\_t tid);}

    \begin{itemize}
        \item Waits for \emph{thread} specified by \emph{tid} to terminate 
        \item Returns the \emph{thread's} \emph{exit} status  
        \item Status information is stored in \emph{region mapper} of \emph{thread} until 
              \emph{waitTid()} is called
    \end{itemize}

\end{frame}

\begin{frame}
    \frametitle{createThread}

    \texttt{L4\_ThreadId\_t createThread();}


    \begin{itemize}
        \item \emph{createThead()} returns \emph{global thread id} of new \emph{thread}.
        \item \emph{Task server} tells \emph{syscall server} to create a new \emph{thread} inside specified \emph{address space}
        \item \emph{Task server} tells \emph{region mapper} to start thread
        \item \emph{Region mapper} creates \emph{thead stack} and sends start message to \emph{thread}
    \end{itemize}

\end{frame}

\begin{frame}
    \frametitle{getThreadStatus}

    \texttt{L4\_Word\_t getThreadStatus(in L4\_ThreadId\_t tid);}

    \begin{itemize}
        \item Uses \emph{schedule()} system call to check whether a \emph{thread} is still alive. Returns the status code which is returned by \emph{schedule}.
    \end{itemize}
\end{frame}

\subsection{Settings, Statistics \& Status}

\begin{frame}
    \frametitle{Settings, Statistics \& Status}

    \begin{itemize}
        \item \emph{setStatisticInterval()}
        \item \emph{setTimeslice()}
        \item \emph{setPriority()}
        \item \emph{setPreemptionDelay()}
    \end{itemize}
\end{frame}

\begin{frame}
    \frametitle{setStatisticInterval}

    \texttt{void setStatisticInterval(in L4\_Word\_t interval);}


    \begin{itemize}
        \item \emph{interval} is specified in milliseconds
        \item Sets interval in which statistics are collected
        \item Uses \emph{schedule} syscall and an ugly hack which uses the total quantum 
    \end{itemize}

\end{frame}

\begin{frame}
    \frametitle{setTimeslice}

    \texttt{void setTimeslice(in L4\_Word\_t timeslice);}

    \begin{itemize}
        \item \emph{timeslice} is specified in milliseconds
        \item Sets length of timeslice
    \end{itemize}

\end{frame}

\begin{frame}
    \frametitle{setPriority}

    \texttt{L4\_Word\_t setPriority(in L4\_Word\_t priority);}

    \begin{itemize}
        \item Sets priority of \emph{thread} identified by \emph{TID} 
    \end{itemize}
\end{frame}

\begin{frame}
    \frametitle{setPreemptionDelay}

    \texttt{L4\_Word\_t setPreemptionDelay(in L4\_ThreadId\_t tid, in L4\_Word\_t sensitivePrio, in L4\_Word\_t maxDelay);}

    \begin{itemize}
        \item Sets preemption delay of \emph{thread} identified by \emph{TID} 
    \end{itemize}
\end{frame}

\section{Statistics}
\subsection{Statistics over virtual Filesystem}

\begin{frame}
    \frametitle{Statistics}

    \begin{itemize}
        \item Collected statistics are accessed via virtual filesystem 
    \end{itemize}
\end{frame}


\begin{frame}
    \frametitle{Questions?}

    \begin{itemize}
        \item Please feel free to ask \emph{questions} or give \emph{comments}!
    \end{itemize}
\end{frame}



\end{document}
